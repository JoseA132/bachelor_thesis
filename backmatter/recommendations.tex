%===================================================================================
% Chapter: Recommendations
%===================================================================================
\Chapter*{Recomendaciones}\label{chapter:recommendations}
%===================================================================================

A pesar de que esta investigación está orientada hacia el descubrimiento de conocimiento en documentos médicos y del idioma español, el modelo de anotación y la ontología propuesta son de propósito general. Esto permite su aplicación en otros dominios e idiomas. Se propone la anotación de corpus de otros dominios en el modelo de anotación definido en este estudio. Al mismo tiempo, tendría gran connotación la creación de grafos de conocimiento mediante la utilización de estos corpus y basados en el modelo ontológico ofrecido.

Se propone comprobar la efectividad de la propuesta de solución ofrecida, pues como se mencionó anteriormente, una deficiencia clara a la hora de calcular las estadísticas expuestas en la tabla \ref{tab:data_driven_evaluation_stats} es que se usó un corpus muy pequeño para crear el grafo de conocimiento y la no existencia de uno independiente pero del mismo dominio y anotado con el formato de modelo propuesto para la posterior validación de la base de conocimiento creada.

La resolución de referencias y correferencias mejoraría en gran medida el descubrimiento de conocimiento implícito en el grafo y debe mejorar los resultados obtenidos. Esta es una tarea que usualmente se intenta resolver usando inteligencia artificial y es un reto que se propone para trabajo futuro, dando continuidad a la línea de investigación presentada en este trabajo.

Otra de las propuestas consideradas es la creación de un grafo de conocimiento a partir de un corpus de dominio específico, siguiendo la línea de investigación de este estudio. A su vez, fomentar el análisis de este corpus en un grupo de expertos en el dominio, y de esta manera corroborar cuán relevante es el conocimiento implícito descubierto a través del grafo resultante en comparación al extraído por los especialistas.

Además se propone la creación de una aplicación para computadoras, móviles y/o páginas web, la cual podría ofrecer sugerencias de enfermedades dados los síntomas especificados por el usuario, y a su vez, posibles tratamientos para las mismas. Esto sería posible mediante la utilización del grafo de conocimiento creado a partir del corpus usado en esta investigación, el cual es de dominio médico.