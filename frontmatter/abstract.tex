%===================================================================================
% Chapter: Abstract (english)
%===================================================================================
\ChapterOutOfToc{Abstract}\label{chapter:abstract_english}
%===================================================================================

\hyphenation{de-fined}
\hyphenation{ex-posed}
\hyphenation{knowl-edge}
\hyphenation{scheme}

In recent years there has been an increase in the development of techniques for automatic knowledge discovery from documents written on natural language. Automatic processing provides the possibility to analyze collections of information containing a large number of texts. In the medical field, the rise of these technologies is significantly special, because it allows taking advantage of the huge amount of data available for it and improve research on this area, which is really important to society. These techniques tend to rely on annotated corpus, and they are a scarce resource. This becomes a critical fact in Spanish language, where the existing amount of them is very low and from a less general domain.

In this study, a general-purpose annotation model is defined to capture the most relevant semantic features contained in text documents. Also, an ontology scheme is presented and used for automatic knowledge extraction. A theoretical step by step implementation of a computational algorithm aiming to build a knowledge graph from an annotated corpus and following the rules of the previously defined ontology is also proposed. Finally, the evaluation and validation process is exposed, as well as statistics results.

The results achieved demonstrate that knowledge discovery constitutes an active research field, where machine learning techniques can be applied achieving positive results. The verification and comparison of a specific knowledge graph built from the proposals provided in this investigation against the learning and interpretation skills of a group of experts on the same field is proposed. Also, the continuation of this research line is suggested, aiming to improve the effectiveness of the proposals given and their application in other domains.