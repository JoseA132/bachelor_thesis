%===================================================================================
% Chapter: Supervisor Opinion
%===================================================================================

\begin{opinion}
	La representación de conocimiento se ha convertido en un área de investigación muy activa en los últimos años, motivada tanto por la disponibilidad masiva de nuevos recursos, como por la necesidad de hacer computacionalmente tratable el volumen de datos producidos diariamente. Su relevancia en tareas más amplias, como el descubrimiento automático de conocimiento, la vuelven un área crucial para el desarrollo de varios sectores de la sociedad. En el dominio médico, la aplicación de estas técnicas se vuelve especialmente interesante, ya que procedimientos de inferencia sobre una base de conocimiento puede potencialmente ayudar a diseñar nuevos tratamientos para combatir enfermedades aún no resueltas. En este marco se desarrolla la tesis de licenciatura de José Ariel Romero Costa, con quien pude trabajar este último año en el diseño y validación de un algoritmo para la construcción de ontologías a partir de textos anotados. Esta tesis da continuidad a una línea de investigación que se ha venido desarrollando en la facultad en los últimos años ligada al descubrimiento de conocimiento.

	La propuesta de José consiste en un algoritmo para la creación automática de ontologías a partir de una colección anotada de documentos. El sistema utiliza el esquema de anotación del \emph{eHealth-KD Challenge} que ha sido empleado en dos competencias internacionales de extracción de conocimiento, en el marco de los eventos \emph{IberLEF 2019} e \emph{IberLEF 2020}. El trabajo conllevó reconstruir un corpus de texto de Medline sobre el que identificar y reordenar las oraciones del corpus anotado. A partir de las entidades y relaciones señaladas en el texto, se realiza un proceso de normalización con el objetivo de unificar aquellas entidades que difieren sintácticamente pero comparten la misma semántica. La tesis presenta un procedimiento para organizar la información recogida en múltiples oraciones, formando una base de conocimiento que integra las distintas instancias de anotaciones mencionadas entre colecciones. La representación final obtenida constituye un paso de avance en la formalización del esquema de anotación, y sienta las bases para futuros procesos de inferencia.

	Durante el desarrollo de la tesis José demostró independencia y creatividad para lidiar con los problemas encontrados. Tuvo que dominar conceptos y tecnologías del estado del arte, con muchas de las cuales no tuvo contacto durante la carrera. Los problemas que hubo de resolver le servirán de aprendizaje para su desarrollo futuro. El proceso de investigación e implementación desarrollado por José queda recogido en un documento de tesis que avala la capacidad adquirida para presentar resultados de investigación de forma concisa y coherente. Todo esto lo han realizado a la par de las actividades docentes, como estudiante de pregrado y como alumno ayudante de la asignatura \emph{Programación}, donde ha sabido asumir con éxito todas las responsabilidades y retos.

	José ha sido alumno ayudante desde su tercer año en la carrera, tiempo que pude compartir con él directamente en clases y en las reuniones del colectivo. En esos años he podido comprobar su interés y dedicación por la asignatura y otros temas relacionados. Este último ejercicio demuestra que ya ha adquirido la madurez necesaria para desarrollar proyectos de alta complejidad con calidad y esmero. Como tutor, estoy complacido por los resultados obtenidos, y por el trabajo realizado con José, que aunque no estuvo exento de obstáculos, logró superar los desafíos. Por estos motivos estoy convencido de que José será un excelente profesional de la Ciencia de la Computación.

	\vspace{1cm}
	\begin{flushright}
		\emph{MSc. Juan Pablo Consuegra Ayala}\\
		Facultad de Matemática y Computación\\
		Universidad de La Habana
	\end{flushright}
\end{opinion}