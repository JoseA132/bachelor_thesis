%===================================================================================
% Chapter: Thanks
%===================================================================================

\begin{acknowledgements}
	Considero fuertemente que, si un día no hubiera conocido a una persona en específico o algo \doublequote{tan sencillo} como no haber saludado a alguien o no haber visto una noticia o información, hoy, quizás, no sería quien soy. Por este motivo aprovecho la investigación que aquí se presenta para agradecer a todas las personas que me han llevado a ser quien soy hoy. Hayamos compartido momentos buenos, malos o un simple gesto de saludo. Gracias a ellos siempre quedarán las marcas que me hicieron llegar a este punto, forjando mi carácter, físico y cualidades buenas y malas.
	
	Algunos de ellos me acompañan desde que nací, por ejemplo, mis padres Ramón y Amparo, que me han dedicado gran parte de su vida y por lo que estoy eternamente agradecido. No tengo manera de pagar todo lo que han hecho por mí, más allá de que con cada paso que doy intento que se sientan más orgullosos de mí. Mi hermano Julio y su esposa Judith, mi abuela Rosa, a la cual cariñosamente desde pequeño le apodé Ia y así la conocen hoy día, a mis tíos Teté y Alberto, mis primos Mayté y Albertico, mis padrinos María Eugenia y Manero. En fin, sin ánimos de extenderme tanto agradezco a modo general a mí familia, sin importar dónde vivan, por haberme apoyado y ayudado en las decisiones que tomé y al mismo tiempo, ofrecer consejos y su amor incondicional.
	
	Otros aparecieron en mi vida, pero lo importante no es cuándo lo hayan
	hecho, sino los aportes que en ella hicieron y que hayan llegado para quedarse. Agradezco a mi novia Odalmis por todo el apoyo y amor que me ha dado, a Ismael, gran amigo que me ayudó durante los cinco años de universidad en los viajes de Matanzas a La Habana y viceversa para que pudiera estudiar y alimentarme, e incluso, me dio varios consejos como si fuera mi segundo padre. A Mayte y Jorge, junto a Fany y Jorge Carlos mis vecinos, quienes en estos últimos dos años han sido como mis padres y hermanos adoptivos en La Habana. A Carlos mi padrino de boda, del cual estoy muy agradecido de haber tenido la posibilidad de conocer y entablar una buena amistad. Nardo, quien nos ha ayudado a mí y a mi familia a lo largo de todos estos años.
	
	Otras personas han quedado en el pasado, pero no porque nos hayamos distanciado emocionalmente, sino físicamente, las distintas situaciones en las que nos puso la vida nos llevó a dejar de vernos, pero son personas que cada vez que hablamos o nos vemos recordamos los momentos compartidos con gran alegría y vivimos nuevos para recordarlos en la próxima ocasión. Ellos son Andy, mi amigo de la infancia, crecimos juntos y fue como un hermano para mí. Arla, su abuela, la cual también me cuidó y enseñó como si fuera su nieto. Yusmeidys, amiga de la vocacional, vivimos muy buenos momentos juntos y entablamos una bonita amistad, si de algo es culpable, es de haber alimentado mi \doublequote{bichito interior} con los deseos de estudiar medicina, pero lo logró tarde, una vez comenzado a cursar esta carrera, aunque por ese motivo trabajaré duro en el futuro para dedicarme a la medicina computacional. A Yudisleydis, mi profesora de la primaria, que me inculcó los primeros pasos en el mundo del estudio, me enseño a leer, escribir y calcular, y fue de las primeras personas que incentivaron las matemáticas en mí.
	
	Por último, pero no menos importante, dos seres que a pesar de que no pueden hablar, no hacen más que expresar sus sentimientos y amor hacia mí: mis hijos caninos Gema y Ody, los cuales me acompañan desde hace $4$ y $3$ años respectivamente. En más de una ocasión han sabido alegrarme el día en una situación donde me sentía triste y estoy más que agradecido y orgulloso por como son.
	
	Este último reto no lo llevé a cabo solo, por eso quiero agradecer también al profesor Juan Pablo por tutorear y ayudarme en esta investigación. Quisiera darle mención a todos pero no puedo, son muchos los que han aparecido y estado presente en mi vida. Espero que sigamos relacionándonos en el futuro, junto con las nuevas personas que conoceré.
	
	\begin{phrase}[3in]
		Indudablemente hoy cierro un capítulo en mi vida: uno que llevo escribiendo durante los últimos $19$ años. Gracias a todo aquel que aportó un granito de arena en mi formación y que creyó en mí; incluso en los momentos en que ni yo mismo creía.
	\end{phrase}
	
	\begin{flushright}
		José Ariel Romero Costa \\
		Facultad de Matemática y Computación \\
		Universidad de La Habana
	\end{flushright}
\end{acknowledgements}