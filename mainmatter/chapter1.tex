%===================================================================================
% Chapter: Automatic Generation of Ontologies
%===================================================================================
\Chapter{Generación Automática de Ontologías}\label{chapter:automatic_generation_of_ontologies}
%===================================================================================

En los últimos años, el desarrollo de especificaciones formales explícitas de los términos en el dominio y las relaciones entre ellos~\cite{ref:4} en una ontología ha dejado de tener lugar en laboratorios o departamentos de inteligencia artificial para pasar a tener un rol protagónico en los escritorios de expertos en el tema.

Las ontologías se han vuelto común en el espacio de la \textit{red mundial} (conocida en inglés como \textit{world wide web}). En esta red ellas abarcan un gran espectro de campos, desde grandes taxonomías que categorizan sitios web, como sucede en Yahoo, hasta categorizaciones de productos a la venta y sus características, como sucede en Amazon.

El \textit{World Wide Web Consortium} (W3C) desarrolló el \textit{Resource Description Framework}~\cite{ref:5} (RDF, traducido al español como \textit{Marco de Descripción de Recursos}). El mismo, es un lenguaje para codificar el conocimiento en las páginas web y que sea comprensible para los usuarios que buscan esa información.

La \textit{Agencia de Proyectos de Investigación Avanzada de Defensa} (del inglés \textit{Defense Advanced Research Projects Agency}, DARPA), en conjunto con W3C, desarrollaron el \textit{DARPA Agent Markup Language} (DAML, traducido al español como \textit{Lenguaje de Marcado de DARPA para Agentes}), el cual es una extensión de RDF con construcciones más expresivas destinadas a facilitar la interacción de los agentes en la web.~\cite{ref:6}

Muchas disciplinas desarrollan ontologías estandarizadas que los expertos en el dominio pueden usar para compartir y anotar información en sus campos. La medicina, por ejemplo, ha producido vocabularios estructurados, estandarizados y extensos como \textit{SNOMED}~\cite{ref:7} y la red semántica del \textit{Unified Medical Language System}~\cite{ref:8} (traducido al español como \textit{Sistema de Lenguaje Médico Unificado}).

También están surgiendo amplias ontologías de propósito general. Por ejemplo, \textit{United Nations Development Program} (traducido al español como \textit{Programa de las Naciones Unidas para el Desarrollo}) y \textit{Dun \& Bradstreet} juntaron sus esfuerzos para desarrollar la ontología \textit{United Nations Standard Products and Services Code}~\cite{ref:9} (UNSPSC, traducido al español como \textit{Código Estándar de Productos y Servicios de las Naciones Unidas}) que proporciona terminología para productos y servicios.

Una ontología define un vocabulario común para los investigadores que necesitan compartir información en un determinado dominio, y a la misma vez, facilita la búsqueda y comprensión de esta información por personas no expertas en el tema. Esto incluye definiciones computacionalmente interpretables de conceptos básicos y relaciones entre ellos pertenecientes al dominio. Hay una interrogante que cabe preguntarse, ¿por qué alguien quisiera desarrollar una ontología? Algunas de las razones principales son:

\begin{itemize}
	\item[•] Compartir conocimiento de la estructura de la información entre investigadores y/o usuarios.
	\item[•] Permitir la reutilización del conocimiento del dominio.
	\item[•] Hacer explícitas las suposiciones o conocimientos a priori del dominio.
	\item[•] Separar el conocimiento explícito del dominio del conocimiento implícito operacional.
	\item[•] Analizar el conocimiento del dominio.
\end{itemize}

\textit{Compartir conocimiento de la estructura de la información entre investigadores y/o usuarios} es uno de los objetivos comunes en el desarrollo de ontologías~\cite{ref:10,ref:4}. Por ejemplo, si varios sitios web diferentes entre sí contienen información médica o proporcionan servicios médicos de comercio electrónico, y estos comparten y publican la misma ontología subyacente de los términos que utilizan, los agentes informáticos pueden extraer y agregar información de los mismos. Además, estos últimos pudieran utilizar dicha información para responder consultas de los usuarios o como datos de entrada para otras aplicaciones.

\textit{Permitir la reutilización del conocimiento del dominio} fue una de las fuerzas impulsoras detrás del reciente aumento de la investigación ontológica. Por ejemplo, los modelos para muchas áreas diferentes deben representar la idea de tiempo. Esta representación incluye, entre otros, las nociones de intervalos, puntos y medidas relativas a este. Si un grupo de investigadores desarrolla tal ontología en detalle, otros pueden simplemente reutilizarla para sus dominios. Además, si se necesita construir una grande, se pueden integrar varias ya existentes que describan partes específicas de la rama deseada. También se puede reutilizar una de propósito general, como UNSPSC, y extenderla para describir el área de interés.

\textit{Hacer explícitas las suposiciones o conocimientos a priori del dominio} hace posible cambiarlas fácilmente si cambian las ideas tenidas de antemano en este tema. Los supuestos de \textit{codificación rígida} (del inglés \textit{hard-coding}) sobre el mundo hechos en lenguajes de programación hacen que estas no solo sean difíciles de encontrar y comprender, sino también de cambiar, en particular para alguien sin experiencia en el ámbito computacional. Además, las especificaciones explícitas del conocimiento del dominio son útiles para los nuevos usuarios que deben aprender qué significan los términos de este.

\textit{Separar el conocimiento explícito del dominio del conocimiento implícito operacional} es otro uso común de las ontologías. Se puede describir la tarea de configurar un producto a partir de sus componentes, de acuerdo con una especificación requerida e implementar un programa que realice esta configuración independientemente del producto.~\cite{ref:11} Seguido de esto, se puede desarrollar una ontología de componentes y características de los ordenadores personales y aplicar el algoritmo para configurar uno de ellos a medida. También este último puede usarse para realizar la misma tarea en ascensores, si se le \doublequote{alimenta} con una ontología de los elementos de estos.~\cite{ref:12}

\textit{Analizar el conocimiento del dominio} es posible una vez que se dispone de una especificación declarativa de los términos. El análisis formal de estos es extremadamente valioso cuando se intenta reutilizar ontologías existentes y ampliarlas.~\cite{ref:13}