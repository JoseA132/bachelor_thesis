%===================================================================================
% Chapter: Proposed Solution
%===================================================================================
\chapter{Propuesta de Solución}\label{chapter:proposed-solution}
Esta investigación busca poder expresar un corpus anotado a través de una ontología definida, generando un grafo de conocimiento como resultado. Otro de los objetivos claros, es poder hacer esto mediante un algoritmo computacional, el cual debe ser, además, finito y determinista. 

%===================================================================================

\section{Primeros pasos}
Primeramente debemos hacer algunas definiciones sobre las cuales nos basaremos en lo adelante. Estas son {\it algoritmo}, {\it algoritmo computacional}, {\it algoritmo finito} y {\it algoritmo determinista}.

\begin{definition}
	Un \textbf{algoritmo} es un conjunto finito y ordenado de instrucciones bien definidas y estructuradas que nos permite llevar a cabo una tarea o encontrar la solución a un determinado problemas.
\end{definition}

\begin{definition}
	Un \textbf{algoritmo computacional} es un algoritmo que puede ser programado en una máquina de cómputo.
\end{definition}

\begin{definition}
	Un \textbf{algoritmo finito} es un algoritmo que termina luego de ejecutar una cantidad finita de instrucciones.
\end{definition}

\begin{definition}
	Un \textbf{algoritmo determinista} es un algoritmo finito que produce el mismo resultado siempre que tenga la misma entrada. Además, la \guillemot{\texttt{máquina interna}} pasará siempre por la misma secuencia de estados, o un poco más informal, la secuencia de instrucciones y resultados de las mismas será siempre igual.
\end{definition}

\section{Analizador sintáctico}

\section{Construcción del grafo}

\section{}