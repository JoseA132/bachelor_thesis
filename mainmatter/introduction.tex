%===================================================================================
% Chapter: Introduction
%===================================================================================
\Chapter*{Introducción}\label{chapter:introduction}
%===================================================================================

\hyphenation{herra-mien-tas}
\hyphenation{corre-la-cio-nar-los}

El desarrollo tecnológico se ha exacerbado con el advenimiento cada vez mayor del uso del Internet y otros medios avanzados y efectivos que garantizan un mejor futuro para cuestiones importantes de la vida. Debido al continuo aumento del flujo informativo, se hace cada vez más necesaria la utilización de herramientas que permitan identificar, capturar y representar el conocimiento dentro de los sistemas de información, ya sea de dominio específico o de propósito general.

Para ello, ciencias avanzadas como la Ciencia de la Computación y de la Comunicación comprenden la ontología como la definición de conceptos y relaciones en algún dominio, de forma compartida y consensuada. Esta conceptualización debe ser representada de una manera formal, legible y utilizable por los ordenadores~\cite{ref:30}. Son creadas para limitar la complejidad de cualquier tema y para organizar la información. Es una medida eficaz en las soluciones de problemas comunes en la vida diaria, que debido a la sobreinformación no se podrían llevar a cabo de forma manual.

Otro de sus beneficios es el hecho de que permiten crear entendimiento compartido al unificar los diferentes puntos de vista. Esto sirve para facilitar la comunicación entre los actores implicados en la construcción de sistemas de información referidos al dominio. Además, permiten el reuso del conocimiento del tema, pues sirve de base ya creada para posteriores investigaciones. También facilitan la recuperación, integración e interoperatividad entre fuentes de conocimiento heterogéneas. Con ellas se provee una base para la representación del conocimiento del dominio y ayudan a identificar las categorías semánticas del mismo~\cite{ref:1}.

Surge entonces el creciente interés de estudiar técnicas para el descubrimiento automático de conocimiento. El procesamiento automático trae consigo la posibilidad de analizar colecciones masivas de información. Sin embargo, la mayor parte de estas colecciones almacenan la información disponible en documentos textuales escritos en lenguaje natural. La naturaleza en que se expresa la información y su estructura semántica poco unificada se vuelven la principal fuente de retos de dicho procesamiento.

En la actualidad, las ontologías se están aplicando en áreas heterogéneas. Entre ellas se encuentran la búsqueda de información, el comercio electrónico, configuraciones de aplicaciones o productos. Además, todo sitio web grande debería usarlas, al menos para organización y navegación~\cite{ref:31}. También se están utilizando para el desarrollo de mecanismos que faciliten la comunicación entre las personas y las máquinas por medio del lenguaje natural~\cite{ref:2}.

En el contexto de la salud y la medicina las ontologías adquieren particular interés, debido a que se están utilizando cada vez más para la solución de disímiles tareas, como la recuperación de información y la búsqueda de respuestas en fragmentos de texto que resuelven preguntas. Diariamente se publican muchos artículos médicos y se hace imposible acceder a todos y obtener conocimiento sobre las novedades médicas y las herramientas que se van desarrollando para solucionar las enfermedades o los problemas de la salud de manera general.

La extracción automática de conocimiento proveería de una herramienta para asistir el desarrollo en esta área a partir de la normalización e integración de los resultados encontrados. Una vez extraído y representado el conocimiento computacionalmente, procesos de inferencia permitirían el descubrimiento de nuevo conocimiento. Ejemplo de esto es el constante descubrimiento de nuevas interacciones entre medicamentos, proteínas y genes. Un sistema de descubrimiento de conocimiento posibilitaría la detección automática de
nuevas relaciones entre ellos, y por ende, el descubrimiento de nuevas causas de enfermedades, síntomas y tratamientos.\\

Algunas de las razones principales para el desarrollo de una ontología son:

\begin{itemize}
	\item[•] Compartir conocimiento de la estructura de la información entre investigadores y/o usuarios.
	\item[•] Permitir la reutilización del conocimiento del dominio.
	\item[•] Hacer explícitas las suposiciones o conocimientos a priori del dominio.
	\item[•] Separar el conocimiento explícito del dominio del conocimiento implícito operacional.
	\item[•] Analizar el conocimiento del dominio.
\end{itemize}

\textit{Compartir conocimiento de la estructura de la información entre investigadores y/o usuarios} es uno de los objetivos comunes en el desarrollo de ontologías~\cite{ref:4,ref:10}. Por ejemplo, si varios sitios web diferentes entre sí contienen información médica o proporcionan servicios médicos de comercio electrónico, y estos comparten y publican la misma ontología subyacente de los términos que utilizan, los agentes informáticos pueden extraer y agregar información de los mismos. Además, estos últimos pudieran utilizar dicha información para responder consultas de los usuarios o como datos de entrada para otras aplicaciones.

\textit{Permitir la reutilización del conocimiento del dominio} fue una de las fuerzas impulsoras detrás del reciente aumento de la investigación ontológica. Por ejemplo, los modelos para muchas áreas diferentes deben representar la idea de tiempo. Esta representación incluye, entre otros, las nociones de intervalos, puntos y medidas relativas a este. Si un grupo de investigadores desarrolla tal ontología en detalle, otros pueden simplemente reutilizarla para sus dominios. Además, si se necesita construir una grande, se pueden integrar varias ya existentes que describan partes específicas de la rama deseada. También se puede reutilizar una de propósito general, como UNSPSC, y extenderla para describir el área de interés.

\textit{Hacer explícitas las suposiciones o conocimientos a priori del dominio} hace posible cambiarlas fácilmente si cambian las ideas tenidas de antemano en este tema. Los supuestos de \textit{codificación rígida} (del inglés \textit{hard-coding}) sobre el mundo hechos en lenguajes de programación hacen que estas no solo sean difíciles de encontrar y comprender, sino también de cambiar, en particular para alguien sin experiencia en el ámbito computacional. Además, las especificaciones explícitas del conocimiento del dominio son útiles para los nuevos usuarios que deben aprender qué significan los términos de este.

\textit{Separar el conocimiento explícito del dominio del conocimiento implícito operacional} es otro uso común de las ontologías. Se puede describir la tarea de configurar un producto a partir de sus componentes, de acuerdo con una especificación requerida e implementar un programa que realice esta configuración independientemente del producto~\cite{ref:11}. Seguido de esto, se puede desarrollar una ontología de componentes y características de los ordenadores personales y aplicar el algoritmo para configurar uno de ellos a medida~\cite{ref:12}.

\textit{Analizar el conocimiento del dominio} es posible una vez se disponga de una especificación declarativa de los términos. El análisis formal de estos es extremadamente valioso cuando se intenta reutilizar ontologías existentes y ampliarlas~\cite{ref:13}.

\noindent\textbf{\large Problemática}

En disímiles ocaciones, es necesaria la lectura de un amplio grupo de documentos con gran cantidad de páginas para poder tener conocimiento acerca de algún tema. Incluso algunas veces la información leída no es relevante para lo que se desea, y por tanto, fue una inversión de tiempo en vano. Las ontologías, por otra parte, aceleran en gran medida este proceso, pues el análisis y representación de uno o más corpus en un grafo de conocimiento es cuestión de segundos. Esto posibilita posteriormente, buscar la información deseada a través de consultas realizadas a un sistema computacional.

Para diseñar una ontología no existe una única forma o metodología correcta a emplear y tampoco es objetivo de este estudio definir una. Con esta investigación se busca dar solución al problema de representar un corpus de documentos anotados como base de conocimiento, por medio de una ontología. Para ello es necesaria la definición de la propia ontología a usar y un algoritmo computacional que posibilite la realización de este proceso.

La generación de ontologías es un proceso que de ser realizado de forma manual, toma demasiado tiempo y esfuerzo. Además, en aras de completar la construcción de una base de conocimientos medianamente buena o buena, es necesario involucrar expertos en el dominio. Por otra parte, estas restricciones imposibilitan la realización de una ontología para todos los posibles corpus de estudio. En cambio, con esta investigación se busca realizar este proceso de forma totalmente automática, sin la necesidad tener conocimiento previo del dominio y tras la espera de unos pocos segundos puede verse el resultado de la ontología construida.

Este problema lleva implícito el procesamiento de lenguaje natural, pues en este están escritos los corpus de documentos que serán usados. Además, el campo de estudio de la generación automática de ontologías es relativamente moderno. Este conjunto de aspectos lo hace ser un problema interesante. Es por esto que es el objetivo de esta investigación.

En esta investigación, además de definir un esquema de ontología de propósito general, se lleva a cabo la realización de una base de conocimiento mediante la utilización de un corpus de dominio médico extraído de \textit{Medline}~\cite{ref:3}.\\

\noindent\textbf{\large Objetivos}

La investigación se plantea como objetivo general definir un diseño de ontología de propósito general que sea capaz de representar el conocimiento descrito en uno o más corpus anotados.

Para darle solución al objetivo general es necesaria la implementación de un algoritmo computacional capaz de representar uno o más corpus de documentos anotados como un grafo de conocimiento que responde a las reglas establecidas por la ontología propuesta.

Se proponen los siguientes objetivos específicos:
\begin{itemize}
	\vspace{-0.27cm}
	\item[•] Estudiar los esquemas de anotación y corpus usados en diversas tareas de extracción del conocimiento.
	\vspace{-0.27cm}
	\item[•] Definir un esquema conceptual de anotación para la representación de los rasgos semánticos más relevantes en textos escritos en lenguaje natural.
	\vspace{-0.27cm}
	\item[•] Definir un formato de anotación de archivos para el esquema conceptual previamente definido.
	\vspace{-0.27cm}
	\item[•] Diseñar una propuesta de ontología donde se pueda representar un corpus de documentos escritos en lenguaje natural.
	\vspace{-0.27cm}
	\item[•] Implementar un algoritmo computacional para representar un corpus anotado como grafo de conocimiento a través de dicha ontología.
	\vspace{-0.27cm}
	\item[•] Implementar un marco experimental para la evaluación de la propuesta de solución.
\end{itemize}

\noindent\textbf{\large Organización de la tesis}

El contenido de la tesis se organiza de la siguiente forma. El capítulo \ref{chapter:automatic_generation_of_ontologies} introduce los principales conceptos relacionados con las ontologías y la extracción y representación de conocimiento. En este capítulo, además, se analizan los principales corpus y representaciones semánticas existentes en la literatura. El capítulo \ref{chapter:annotation_model} describe un modelo de anotación de propósito general que busca capturar los rasgos semánticos más importantes en documentos de texto. En el capítulo \ref{chapter:proposed_solution} se presenta una propuesta para la creación de un grafo de conocimiento a través de un corpus anotado. En el capítulo \ref{chapter:analysis_of_results} se muestran los resultados alcanzados en esta investigación, y en función de estos, se discute la efectividad de cada uno de los elementos propuestos en la misma. La investigación finaliza presentando las conclusiones pertinentes y las recomendaciones para su continuidad y mejora.